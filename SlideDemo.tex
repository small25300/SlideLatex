\documentclass{beamer}
\usepackage[UTF8]{ctexcap}
\usepackage[skins, breakable, theorems,fitting,most]{tcolorbox}% 用来生成各式各样的框
\setCJKfamilyfont{songti}[AutoFakeBold={3}]{SimSun}
\setmainfont{Times New Roman}
\usepackage{xcolor}

\usetheme{AnnArbor}
%\setmainfont{Microsoft YaHei}

\title{ACM大会汇报}
%\subtitle{小试牛刀}
\author{王小龙}
\institute{电信系}
\date{2019年6月20日}
%===================目录格式定义设置==========================%
\newcommand{\beamertemplatenumberedsectiontocMyself}{
	\usetemplatetocsection[20!averagebackgroundcolor]
	{\leavevmode\leftskip=6.5em\vspace{0ex}\color{beamerstructure}\inserttocsectionnumber.\hspace{1em}\inserttocsection\par}
	
	\usetemplatetocsubsection[20!averagebackgroundcolor]
	{\vspace*{2ex}\leavevmode\leftskip=2em\color{green!45!black}\rlap{\hskip-2em\inserttocsectionnumber.\inserttocsubsectionnumber}\inserttocsubsection\par}
}
%=====================end===========================%
\begin{document}
	\begin{frame}[plain]
		\titlepage
	\end{frame}
	%========设置frametitle居中============%
	%定义一个新的tcolorbox环境命令frametitlebox,画出一个文本框
	\newtcolorbox{frametitlebox}{width=10cm, colback = yellow!87!red,colframe=red!5!white}
%	\setbeamercolor{frametitle}{fg=red}
	\setbeamerfont{frametitle}{series=\bfseries}
	\setbeamertemplate{frametitle}
	{	
		\begin{centering}%将整个文本框居中
				\vspace{1ex}
			\begin{frametitlebox}% 调用之前定义的新的tcolorbox环境命令frametitlebox
				\centering\insertframetitle\par
			\end{frametitlebox}
		\end{centering}	
	}	
	%==============end====================%
	
	%================catalog(目录) frame============%
	\setbeamerfont{section in toc}{%修改目录中字的大小等格式
		family=\rmfamily,series=\bfseries,size=\Large}
	\setbeamerfont{subsection in toc}{%修改目录中字的大小等格式
		family=\rmfamily,series=\bfseries,size=\large}
	
	
	\begin{frame}[shrink,squeeze,c]\frametitle{目\quad 录}
		\beamertemplatenumberedsectiontocMyself
		\tableofcontents[hideallsubsections]
	\end{frame}
	%======================end===================%
	
	%========每一个section前面插入一个section目录和subsection===%
	\AtBeginSection[]
	{
		\begin{frame}[shrink,squeeze,c]
			\beamertemplatenumberedsectiontocMyself
			\tableofcontents[currentsection,hideallsubsections]
		\end{frame}
	}
	\AtBeginSubsection[]
	{
		\begin{frame}[shrink,squeeze,c]
			\beamertemplatenumberedsectiontocMyself
			\tableofcontents[sectionstyle=show/hide,subsectionstyle=show/shaded/hide]
		\end{frame}
	}
	%==================end===================%	
	
	\section{第一天学术报告安排}
	\subsection[Feasible Computation]{What are the Limits of Feasible Computation-Leslie Valiant,Havard University}
	
%	\setbeamerfont{items}{family=\sffamily}
%	\setbeamerfont{subitem}{family=\sffamily}
%	\setbeamerfont{subsubitem}{family=\rmfamily}
%	Keys are: family, series, shape, size, parent
	\begin{frame}
		\begin{itemize}
%			\setlength{\itemsep}{0.5cm}
			\setbeamertemplate{itemize items}[square]
			\fontfamily{pbk}\Large
			\item Algorithms and Complexity
			
			\begin{itemize}
%				\setlength{\itemsep}{0.2cm}
				\setbeamertemplate{itemize items}[circle]
				\zihao{-4}
				\item Order in Application Space
				\begin{tcolorbox}[colback = white, colframe = blue!70!red]
					\zihao{5}Computer science is possible because applications are not isolated, but have common themes.(e.g. linear algebra very widely used.)
				\end{tcolorbox}
				
				\item Reductions among problems
				\begin{tcolorbox}[colback = white, colframe = blue!70!red]
					\zihao{5}Such as ploynomial time reductions, holographic transformations.
				\end{tcolorbox}
				\item Clusters of Equivalent Applications
				\begin{tcolorbox}[colback = white, colframe = blue!70!red]
					\zihao{5}Such as NP-complete problems.
				\end{tcolorbox}
			\end{itemize}
			\item  Physics
			\item  Computer Science
		\end{itemize}
	\end{frame}

	\subsection{测试}
	\subsection{成功}
	\subsection{你好}
	\subsection{厉害}
	\subsection{风水}
	\subsection{没有}
	\subsection{哈哈}
	

	\section{第二部分}
	\subsection{条件}
	\begin{frame}
		\begin{tcolorbox}[title=Lower separated] This is the upper part.
			\tcblower
			This is the lower part.
		\end{tcolorbox}
	
		\begin{tcolorbox}[sidebyside,title=Lower separated]
			This is the upper part.
			\tcblower
			This is the lower part.
		\end{tcolorbox}
	\end{frame}	

	\section{第三部分}
	\begin{frame}
		\begin{tcolorbox}[upperbox=invisible,colback=white]
			This is a \textbf{tcolorbox} (but invisible).
			\tcblower
			This is the lower part.\\
			sdkljk
		\end{tcolorbox}
	\end{frame}
	\section{第si部分}
	\section{第wu部分}
	\section{第liu部分}
	\section{第qe部分}
	
\end{document}
