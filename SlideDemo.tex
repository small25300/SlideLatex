\documentclass{beamer}
\usepackage[UTF8]{ctexcap}
\usepackage[skins, breakable, theorems,fitting,most]{tcolorbox}% 用来生成各式各样的框


\usepackage{xcolor}

\usetheme{AnnArbor}


\title{我的第一张幻灯片}
\subtitle{小试牛刀}
\author{王小龙}
\institute{西北师范大学知行学院}
\date{2019年6月4日}
%===================目录格式定义设置==========================%
\newcommand{\beamertemplatenumberedsectiontocMyself}{
	\usetemplatetocsection[20!averagebackgroundcolor]
	{\leavevmode\leftskip=6.5em\vspace{0ex}\color{beamerstructure}\inserttocsectionnumber.\hspace{1em}\inserttocsection\par}
	
	\usetemplatetocsubsection[20!averagebackgroundcolor]
	{\leavevmode\leftskip=2em\color{black}\rlap{\hskip-2em\inserttocsectionnumber.\inserttocsubsectionnumber}\inserttocsubsection\par}
}
%=====================end===========================%
\begin{document}
	\begin{frame}[plain]
		\titlepage
	\end{frame}

	%========设置frametitle居中============%
	%定义一个新的tcolorbox环境命令frametitlebox,画出一个文本框
	\newtcolorbox{frametitlebox}{width=10cm, colback = yellow!87!red,colframe=red!5!white}
	\setbeamercolor{frametitle}{fg=red}
	\setbeamerfont{frametitle}{series=\bfseries}
	\setbeamertemplate{frametitle}
	{	
		\begin{centering}%将整个文本框居中
				\vspace{1ex}
			\begin{frametitlebox}% 调用之前定义的新的tcolorbox环境命令frametitlebox
				\centering\insertframetitle\par
			\end{frametitlebox}
		\end{centering}	
	}	
	%==============end====================%
	
	%================catalog(目录) frame============%
	\setbeamerfont{section in toc}{%修改目录中字的大小等格式
		family=\rmfamily,series=\bfseries,size=\Large}
	\begin{frame}[shrink,squeeze,c]\frametitle{目\quad 录}
		\beamertemplatenumberedsectiontocMyself
		\tableofcontents[hideallsubsections]
	\end{frame}
	%=============================end===================%
	
	%========每一个section前面插入一个section目录=========%
	\AtBeginSection[]
	{
		\begin{frame}
		\beamertemplatenumberedsectiontocMyself
		\tableofcontents[currentsection,hideallsubsections]
	\end{frame}
	}
	%==================end===================%	
	
	\section{第一部分}
	\subsection{部分}
	\begin{frame}
		我的内容
	\end{frame}

	\section{第二部分}
	\subsection{部分}
%		\AtBeginSection[]{
%		\begin{frame}<beamer>
%		\frametitle{\sectiontitle}\tableofcontents[
%		sectionstyle=show/shaded,
%		subsectionstyle=show/show/shaded]
%	\end{frame}
%}
	\begin{frame}
		\begin{tcolorbox}[title=Lower separated] This is the upper part.
			\tcblower
			This is the lower part.
		\end{tcolorbox}
	
		\begin{tcolorbox}[sidebyside,title=Lower separated]
			This is the upper part.
			\tcblower
			This is the lower part.
		\end{tcolorbox}
	\end{frame}	

	\section{第三部分}
	\begin{frame}
		\begin{tcolorbox}[upperbox=invisible,colback=white]
			This is a \textbf{tcolorbox} (but invisible).
			\tcblower
			This is the lower part.\\
			sdkljk
		\end{tcolorbox}
	\end{frame}
	\section{第si部分}
	\section{第wu部分}
	\section{第liu部分}
	\section{第qe部分}
	
\end{document}